
\usepackage{algorithm}
\usepackage{algpseudocode}
\usepackage[square,numbers]{natbib}
\usepackage{bbold}
\usetheme{default}

% Make content that is hidden by pauses "transparent"
\setbeamercovered{transparent}

% --- Slide layout settings ---

% Set line spacing
\renewcommand{\baselinestretch}{1.15}

% Set left and right text margins
\setbeamersize{text margin left=12mm, text margin right=12mm}

% Add slide numbers in bottom right corner
\setbeamertemplate{footline}[frame number]

% Remove navigation symbols
\setbeamertemplate{navigation symbols}{}

% Allow local line spacing changes
\usepackage{setspace}

% Change itemized list bullets to circles
\setbeamertemplate{itemize item}{$\bullet$}
\setbeamertemplate{itemize subitem}{$\circ$}

% --- Color and font settings ---

\usepackage{xcolor}


% Slide title background color
\definecolor{background}{HTML}{ede6d8}

% Slide title text color
\definecolor{titleText}{HTML}{B40404}

% Other possible color schemes

% - Light green/dark green -
%\definecolor{background}{HTML}{e4ede4}
%\definecolor{titleText}{HTML}{2e592f}

% - Light blue/dark blue -
%\definecolor{background}{HTML}{d5d9e8}
%\definecolor{titleText}{HTML}{2d375e}

% - Beige/dark blue -
%\definecolor{background}{HTML}{e8e2d5}
%\definecolor{titleText}{HTML}{2d3375}

% Set colors
\setbeamercolor{frametitle}{bg=background, fg=titleText}
\setbeamercolor{subtitle}{fg=titleText}

% Set font sizes for frame title and subtitle
\setbeamerfont{frametitle}{size=\fontsize{15}{16}}
\setbeamerfont{framesubtitle}{size=\small}

% --- Math/Statistics commands ---

% Add a reference number to a single line of a multi-line equation
% Usage: "\numberthis\label{labelNameHere}" in an align or gather environment
\newcommand\numberthis{\addtocounter{equation}{1}\tag{\theequation}}

% Shortcut for bold text in math mode, e.g. $\b{X}$
\let\b\mathbf

% Shortcut for bold Greek letters, e.g. $\bg{\beta}$
\let\bg\boldsymbol

% Shortcut for calligraphic script, e.g. %\mc{M}$
\let\mc\mathcal

% \mathscr{(letter here)} is sometimes used to denote vector spaces
\usepackage[mathscr]{euscript}

% Convergence: right arrow with optional text on top
% E.g. $\converge[p]$ for converges in probability
\newcommand{\converge}[1][]{\xrightarrow{#1}}

% Weak convergence: harpoon symbol with optional text on top
% E.g. $\wconverge[n\to\infty]$
\newcommand{\wconverge}[1][]{\stackrel{#1}{\rightharpoonup}}

% Equality: equals sign with optional text on top
% E.g. $X \equals[d] Y$ for equality in distribution
\newcommand{\equals}[1][]{\stackrel{\smash{#1}}{=}}

% Normal distribution: arguments are the mean and variance
% E.g. $\normal{\mu}{\sigma}$
\newcommand{\normal}[2]{\mathcal{N}\left(#1,#2\right)}

% Uniform distribution: arguments are the left and right endpoints
% E.g. $\unif{0}{1}$
\newcommand{\unif}[2]{\text{Uniform}(#1,#2)}

% Independent and identically distributed random variables
% E.g. $ X_1,...,X_n \iid \normal{0}{1}$
\newcommand{\iid}{\stackrel{\smash{\text{iid}}}{\sim}}

% Sequences (this shortcut is mostly to reduce finger strain for small hands)
% E.g. to write $\{A_n\}_{n\geq 1}$, do $\bk{A_n}{n\geq 1}$
\newcommand{\bk}[2]{\{#1\}_{#2}}

% Math mode symbols for common sets and spaces. Example usage: $\R$
\newcommand{\R}{\mathbb{R}}	% Real numbers
\newcommand{\C}{\mathbb{C}}	% Complex numbers
\newcommand{\Q}{\mathbb{Q}}	% Rational numbers
\newcommand{\Z}{\mathbb{Z}}	% Integers
\newcommand{\N}{\mathbb{N}}	% Natural numbers
\newcommand{\F}{\mathcal{F}}	% Calligraphic F for a sigma algebra
\newcommand{\El}{\mathcal{L}}	% Calligraphic L, e.g. for L^p spaces

% Math mode symbols for probability
\newcommand{\pr}{\mathbb{P}}	% Probability measure
\newcommand{\E}{\mathbb{E}}	% Expectation, e.g. $\E(X)$
\newcommand{\var}{\text{Var}}	% Variance, e.g. $\var(X)$
\newcommand{\cov}{\text{Cov}}	% Covariance, e.g. $\cov(X,Y)$
\newcommand{\corr}{\text{Corr}}	% Correlation, e.g. $\corr(X,Y)$
\newcommand{\B}{\mathcal{B}}	% Borel sigma-algebra

% Other miscellaneous symbols
\newcommand{\tth}{\text{th}}	% Non-italicized 'th', e.g. $n^\tth$
\newcommand{\Oh}{\mathcal{O}}	% Big-O notation, e.g. $\O(n)$
\newcommand{\1}{\mathds{1}}	% Indicator function, e.g. $\1_A$

% Additional commands for math mode
\DeclareMathOperator*{\argmax}{argmax}	% Argmax, e.g. $\argmax_{x\in[0,1]} f(x)$
\DeclareMathOperator*{\argmin}{argmin}	% Argmin, e.g. $\argmin_{x\in[0,1]} f(x)$
\DeclareMathOperator*{\spann}{Span}	% Span, e.g. $\spann\{X_1,...,X_n\}$
\DeclareMathOperator*{\bias}{Bias}	% Bias, e.g. $\bias(\hat\theta)$
\DeclareMathOperator*{\ran}{ran}		% Range of an operator, e.g. $\ran(T)
\DeclareMathOperator*{\dv}{d\!}		% Non-italicized 'with respect to', e.g. $\int f(x) \dv x$
\DeclareMathOperator*{\diag}{diag}	% Diagonal of a matrix, e.g. $\diag(M)$
\DeclareMathOperator*{\trace}{trace}	% Trace of a matrix, e.g. $\trace(M)$
\DeclareMathOperator*{\supp}{supp}	% Support of a function, e.g., $\supp(f)$

% Encadrement
\newcommand{\angles}[1]{\left\langle #1 \right\rangle} % ⟨⟩
\newcommand{\braces}[1]{\left\lbrace #1 \right\rbrace} % {}
\newcommand{\bracks}[1]{\left\lbrack #1 \right\rbrack} % []
\newcommand{\pars}[1]{\left( #1 \right)}               % ()

