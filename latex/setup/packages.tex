% \usepackage[square,numbers]{natbib}         % Pour la bibliographie
\usepackage[nottoc]{tocbibind}
\usepackage{url}            % Pour citer les adresses web
\usepackage[hidelinks]{hyperref}       % Pour activer les liens cliquables
\usepackage[T1]{fontenc}    % Encodage des accents
\usepackage[utf8]{inputenc} % Lui aussi
\usepackage[english]{babel} % Pour la traduction française
\usepackage{numprint}       % Histoire que les chiffres soient bien
\usepackage{eurosym}        % Permet l'utilisation du signe € : \EUR{\num{299792458.38}}
\usepackage{appendix}       % Pour avoir des appendices
\usepackage{xspace}         % Utile lors de la création d'abréviations
\usepackage[dvipsnames]{xcolor}         % Permet de gérer facilement les couleurs.
\usepackage{verbatim}       % Permet d'insérer du code LaTeX sans qu'il soit compilé


% Setup du headers
\usepackage{fancyhdr}       % Fancy headers (page et titre de la section en haut)

\usepackage{etoolbox}


% Package pour avoir la section Annexes
\usepackage{appendix}
\renewcommand{\appendixtocname}{Appendix}  % On renomme dans la table des matières
\renewcommand{\appendixpagename}{Appendix} % On renomme dans le document

% Packages pour images et graphiques
\usepackage{graphicx}   % Inclusion des graphiques
\usepackage{graphics}
\usepackage{wrapfig}    % Dessins dans le texte.
\usepackage[justification=centering]{caption}    % Permet de personnaliser les styles des légendes.
\usepackage{subcaption} % Permet d’insérer au sein d’une figure des sous-figures, chacune d’entre elles disposant d’une légende, en plus de la légende principale.

% Un package pour les dessins (utilisé pour l'environnement {code})
\usepackage{tikz}
\usetikzlibrary{shapes.geometric, arrows}
\usepackage[framemethod=TikZ]{mdframed}

% Packages pour la création de nouvelles commandes
\usepackage{xifthen}
\usepackage{xargs}

% Packages pour la typographie
\usepackage[maxlevel=3]{csquotes} % Propose des commandes pour les citations. L'option maxlevel permet de déterminer le nombre de niveaux d'imbrication maximal.
% Exemple :  \enquote{Lorsque yyy déclare \enquote{zzz} il ne déclare rien du tout}. Les citations en bloc sont également possible : \begin{quote} citation \end{quote}. Pour les citations plus longues, on préférera \begin{quotation} citation longue \end{quotation}.
\usepackage{setspace}             % Permet de modifier l'espacement entre les lignes.


% Packages pour les mathématiques
\usepackage{amsmath}        % La base pour les maths
\usepackage{mathrsfs}       % Quelques symboles supplémentaires
\usepackage{amssymb}        % encore des symboles.
\usepackage{amsfonts}       % Des fontes, eg pour \mathbb.
\usepackage{mathtools}      % Permet d'utiliser des symboles d'égalités spéciaux
% \usepackage{dsfont}         % Pour avoir les notations nécessaires aux ensembles (nécessite mathbb)
% \usepackage{stmaryrd}       % Permet de faire des intervalles avec des doubles barres (intervalles d'entiers)
% \usepackage{esvect}         % Donne plein de possibilité pour les flèches des vecteurs
% \usepackage{systeme}        % Permet de créer facilement des systèmes (en mode mathématique ou en texte)

\usepackage{tcolorbox}
\usepackage{enumitem}
% \usepackage{witharrows}