% !TeX root = main.tex

\documentclass{article}


% if you need to pass options to natbib, use, e.g.:
%     \PassOptionsToPackage{numbers, compress}{natbib}
\usepackage[final]{setup/neurips_2022}

% \usepackage[square,numbers]{natbib}         % Pour la bibliographie
\usepackage[nottoc]{tocbibind}
\usepackage{url}            % Pour citer les adresses web
\usepackage[hidelinks]{hyperref}       % Pour activer les liens cliquables
\usepackage[T1]{fontenc}    % Encodage des accents
\usepackage[utf8]{inputenc} % Lui aussi
\usepackage[english]{babel} % Pour la traduction française
\usepackage{numprint}       % Histoire que les chiffres soient bien
\usepackage{eurosym}        % Permet l'utilisation du signe € : \EUR{\num{299792458.38}}
\usepackage{appendix}       % Pour avoir des appendices
\usepackage{xspace}         % Utile lors de la création d'abréviations
\usepackage[dvipsnames]{xcolor}         % Permet de gérer facilement les couleurs.
\usepackage{verbatim}       % Permet d'insérer du code LaTeX sans qu'il soit compilé


% Setup du headers
\usepackage{fancyhdr}       % Fancy headers (page et titre de la section en haut)

\usepackage{etoolbox}


% Package pour avoir la section Annexes
\usepackage{appendix}
\renewcommand{\appendixtocname}{Appendix}  % On renomme dans la table des matières
\renewcommand{\appendixpagename}{Appendix} % On renomme dans le document

% Packages pour images et graphiques
\usepackage{graphicx}   % Inclusion des graphiques
\usepackage{graphics}
\usepackage{wrapfig}    % Dessins dans le texte.
\usepackage[justification=centering]{caption}    % Permet de personnaliser les styles des légendes.
\usepackage{subcaption} % Permet d’insérer au sein d’une figure des sous-figures, chacune d’entre elles disposant d’une légende, en plus de la légende principale.

% Un package pour les dessins (utilisé pour l'environnement {code})
\usepackage{tikz}
\usetikzlibrary{shapes.geometric, arrows}
\usepackage[framemethod=TikZ]{mdframed}

% Packages pour la création de nouvelles commandes
\usepackage{xifthen}
\usepackage{xargs}

% Packages pour la typographie
\usepackage[maxlevel=3]{csquotes} % Propose des commandes pour les citations. L'option maxlevel permet de déterminer le nombre de niveaux d'imbrication maximal.
% Exemple :  \enquote{Lorsque yyy déclare \enquote{zzz} il ne déclare rien du tout}. Les citations en bloc sont également possible : \begin{quote} citation \end{quote}. Pour les citations plus longues, on préférera \begin{quotation} citation longue \end{quotation}.
\usepackage{setspace}             % Permet de modifier l'espacement entre les lignes.


% Packages pour les mathématiques
\usepackage{amsmath}        % La base pour les maths
\usepackage{mathrsfs}       % Quelques symboles supplémentaires
\usepackage{amssymb}        % encore des symboles.
\usepackage{amsfonts}       % Des fontes, eg pour \mathbb.
\usepackage{mathtools}      % Permet d'utiliser des symboles d'égalités spéciaux
% \usepackage{dsfont}         % Pour avoir les notations nécessaires aux ensembles (nécessite mathbb)
% \usepackage{stmaryrd}       % Permet de faire des intervalles avec des doubles barres (intervalles d'entiers)
% \usepackage{esvect}         % Donne plein de possibilité pour les flèches des vecteurs
% \usepackage{systeme}        % Permet de créer facilement des systèmes (en mode mathématique ou en texte)

\usepackage{tcolorbox}
\usepackage{enumitem}
% \usepackage{witharrows}
\newcommand{\Zt}{\ensuremath{\textbf{Z}_t}\xspace}
\input{setup/math.tex}

\bibliographystyle{setup/acl_natbib}

\title{Likelihood training of Schrödinger bridge using Forward-backward SDEs theory}


\author{
  Benjamin Maurel\\
  MVA - ENSAE\\
  \texttt{benjamin.maurel@ensae.fr}\\
  \And
  Jérémie Stym-Popper\\
  MVA - ENSAE\\
  \texttt{jeremie.stym-popper@ensae.fr}\\
  \And
  Gabriel Watkinson\\
  MVA - ENSAE\\
  \texttt{gabriel.watkinson@ensae.fr}\\
}


\begin{document}

\maketitle

% Possible questions:
% What are the benefits of this formulation over classical diffusion models?
% What are the links between this approach and stochastic control? How can we leverage this link?
% Can this framework be extended to solve any quasi-parabolic PDEs? What are the potential applications?

% \begin{abstract}
% \end{abstract}

\section*{Introduction}

Scored-based Generative Models (SGMs) have gained significant attention in the field of deep image generation.
Since their introduction by \citet{song2020score}, SGMs have emerged as the state-of-the-art approach for image generation.
Unlike likelihood-based models such as VAEs or normalizing flows, which learn the probability density function directly, or adversarial-based models like GANs.
SGMs learn a mapping between a simple distribution, typically Gaussian, and complex, high-dimensional, and often intractable data distributions, such as images.
These models offer several advantages over existing approaches, including high-quality sample generation comparable to GANs without the instability of adversarial training, flexible model architectures, exact log-likelihood computation, and the ability to solve inverse problems without re-training models.
However, SGMs also suffer from pitfalls that limit their applicability, such as difficult training procedure, the impossibility to use a complex distribution as the prior, the fact that the drift can only be linear or degenerate, and slow training and sampling processes compared to other methods.

To address these limitations, recent works by \cite{debortoli2023diffusion, pmlr-v139-wang21l, vargas2021solving} have proposed a new class of generative models inspired by Optimal Transport (OT) theory, specifically using Schrödinger Bridges (SB) \cite{schrodinger1932theorie}.
SBs, similar to SGMs, aim to learn a mapping between two distributions in a finite horizon.
However, SBs offer greater flexibility by allowing the mapping between arbitrary distributions, making them more appealing.
The relationship between SBs and SGMs was not clearly established, as SGMs rely on Stochastic Differential Equations (SDEs), whereas SBs are based on Partial Differential Equations (PDEs).

In this paper, \cite{chen2023likelihood} bridges the gap between these seemingly distinct mathematical concepts by drawing inspiration from the field of Stochastic Optimal Control and utilizing Forward-Backward Stochastic Differential Equations (FBSDEs).
The authors propose a method that enables the solution of PDEs with SDEs, offering computational efficiency and improved precision compared to solving PDEs directly.
This approach introduces a novel training method for SBs, resembling the framework used in flow-based diffusion models.
Additionally, the authors also use a more traditional Iterative Proportional Fitting approach (IPF; \cite{kullback1968probability,debortoli2023diffusion}).

In this report, we will first introduce the two classical methods, Scored-based Generative Models and Schrödinger Bridges, before exploring the Forward-Backward Stochastic Differential Equations method that establishes a connection between the two.
Finally, we will discuss further extensions of this paper, including an application of the method with the Hamilton-Jacobi-Bellman PDE.


% Recently, Scored-based Generative Model (SGM) have received a lot of attention in the field of deep image generation.
% After their introduction by \citet{song2020score}, SGMs have become the state of the art in image generation.
% They have supplanted likelihood-based models, such as VAEs or normalizing flows, which directly learn the distribution's probability density function via maximum likelihood, or adversarial based models, such as GANs.
% SGMs have several important advantages over existing model families: GAN-level sample quality without the instable adversarial training, flexible model architectures, exact log-likelihood computation, and inverse problem solving without re-training models.
% However, SGMs also suffer from a few major pitfalls: they are difficult to train, the prior distribution has to be simple, the drift cannot be complex, and they are slow to train and sample from.

% To try and address these limitations, \cite{debortoli2023diffusion, pmlr-v139-wang21l, vargas2021solving} have proposed a new class of generative model inspired by Optimal Transport (OT) theory using Schrödinger Bridges (SB; \cite{schrodinger1932theorie}).
% Similarly to SGMs, the objective of SBs is to learn a mapping between two distributions in finite horizon, but, unlike SGMs, SBs are capable to map two arbitrary distributions, which makes them more flexible, thus more attractive.
% However, the relation between SBs and SGMs was not clearly established, as SGMs rely on Stochastic Differential Equations (SDEs) whereas SBs are based on Partial Differential Equations (PDEs).

% \cite{chen2023likelihood} breaches the gap between those two seemingly different mathematical objects, inspired by the field of Stochastic Optimal Control, and using Forward-Backward Stochastic Differential Equations (FBSDEs). Eventually, it proposes a method that enables to solve PDEs with SDEs, which is computationally more efficient and more precise than solving PDEs.
% This vision naturally introduces a new way to train SBs that resemble the joint optimization for diffusion flow-based models.
% The authors also propose a more traditional Iterative Proportional Fitting approach (IPF; \cite{kullback1968probability,debortoli2023diffusion}).

% The two classical methods (SGM and SB) will be presented before studying in a third time the FBSDE method that enables to link the two methods. Finally, we will give further extensions to this paper presenting an application of the method with the Hamilton-Jacobi-Bellman PDE.



\section{Some reminders about Score-based Generative Model}

Given a dataset $\braces{\mathbf{x}_1, \mathbf{x}_2, \dots, \mathbf{x}_N}$ of $N$ samples in $\mathbb{R}^d$, the goal of generative models is to learn the underlying data distribution $p_\mathrm{data}$, as to be able to generate new samples from this distribution.
The go to approach for this kind of problem is to use maximum likelihood estimation (MLE) to learn the parameters of a parametric model $p_\theta$ that approximates the distribution. However, this relies on the assumption that $p_\theta$ be a normalized density function. To do so, we have to calculate the normalizing constant, which is intractable in high dimension. Therefore, to train such models using MLE, we either have to restrict the architectures, e.g. to invertible networks in flow models, or approximate the normalizing constant, using variational inference in VAEs for instance. This is not ideal, as it limits the flexibility of the model and the quality of the approximation.

Another solution is to use the score function $s_\theta(x) = \nabla_x \log p_\theta(x)$ instead of the density function $p_\theta$.
This makes the normalization constant disappear thanks to the gradient, allowing a wider range of architectures.
Score matching methods can then be used to learn the score function $s_\theta$, followed Langevin dynamics to sample from the learned score function, giving access to the wanted distribution.
However, this naive approach doesn't work well in practice, especially in low density regions.

The solution is to add a diffusion process to the score matching method, which is the idea behind Score-based Generative Models (SGM).
In SGMs, inputs are "diffused" toward a random noise.
This diffusion process is characterized by a stochastic differential equation (SDE):
\begin{equation}
    \dint{\mathbf{X}_t} = f(t,\mathbf{X}_t)\dint{t} + g(t) \dint{\mathbf{W}_t}, \quad \mathbf{X}_0 \sim p_\mathrm{data}
    \label{eq:forward_sde_sgm}
\end{equation}
where $f(\cdot, t) : \R^n \rightarrow \R^n$, $g(t) \in \R$ are the drift and diffusion coefficients, and $\mathbf{W}_t \in \R^n$ is a standard Wiener process.
For a large time step, we have that $p_T \approx p_\mathrm{prior}$ that closely ressemble a Gaussian distribution, which is easy to sample from.

The associated backward SDE is:
\begin{equation}
    \dint{\mathbf{X}_t} = \bracks{f(t,\mathbf{X}_t) - g(t)^2\nabla_x\log p_t(\mathbf{X}_t)}\dint{t} + g(t) \dint{\mathbf{W}_t}, \quad \mathbf{X}_T \sim p_T
    \label{eq:backward_sde_sgm}
\end{equation}

The score function appears naturally in the backward SDE, which is the reason why SGMs are called score-based. In the case where the drift is simple (e.g. linear or degenerate), it is possible to learn a score network $s(t, x; \theta)$ by regressing the outputs to the ground truth, i.e $\esp{\lambda(t) \norme{s(t, x; \theta) - \nabla_x \log p_{t|x_0}}^2}$, where the expectation is taken over the forward SDE.
$\lambda(t)$ is a weighting function that is crucial in practice but hard to choose well.

More recently, \citet{song2021maximum} have found a Evidence Lower bound (ELBO) that lower-bounds the log-likelihood of the SGM:
\begin{equation}
\log p_0^\mathrm{SGM}(\mathbf{x}_0)
\ge
\esp{\log p_T(\mathbf{X}_t)}
- \int_0^T
\esp{\frac{1}{2}g^2\norme{\mathbf{s}_t}^2
- \nabla_x \cdot (g^2 s_t - f)}
\dint t
\end{equation}
This can be obtained bounding the KL divergence by score-matching loss, where the weighting function $\lambda(t) = g(t)^2$.
This lower bound is important since it will be one of the element that links together the SGM model and the Schrödinger-Bridge problem.

Once the model is trained, we can just substitute the score function by the learned one in the backward SDE \eqref{eq:backward_sde_sgm} to sample from the data distribution.




\section{Schrodinger Bridge}

Unlike SGMs, Schrödinger Bridges (SB) are not generative models, but rather a way to interpolate between two distributions $\mu_0$ and $\mu_1$.
There were first introduced by \citet{schrodinger1932theorie} and are related to the field of regularized optimal transport.
The Schrödinger Bridge problem can be defined as follows:
\begin{equation*}
    \Pi^\star = \argmin\braces{\mathrm{KL}(\Pi|\mathbb{P}) \, | \, \Pi \in \mathcal{P}(\mathcal{C}), \, \Pi_0 = p_{\textrm{data}}, \, \Pi_T = p_{\textrm{prior}}}
\end{equation*}
where $\Pi^0$ is the reference measure, which is set to the path measure of the forward diffusion process in \eqref{eq:forward_sde_sgm}.

The \textbf{Benamou-Brenier} reformulation of the entropy-regularized optimal transport problem gives the following partial differential equations (PDEs):
\begin{equation}
    \label{eq:sb_pdes}
    \begin{cases}
    \frac{\partial \Psi}{\partial t} = -\nabla_x \Psi^\top f - \frac{1}{2}\mathrm{Tr}(g^2 \nabla_x^2 \Psi) \\
    \frac{\partial \widehat{\Psi}}{\partial t} = -\nabla_x \cdot (\widehat{\Psi} f) + \frac{1}{2}\mathrm{Tr}(g^2 \nabla_x^2 \widehat{\Psi})
    \end{cases}
    \mathrm{s.t.}
    \begin{array}{l}
        \Psi(0, \cdot) \widehat{\Psi}(0, \cdot) = p_\mathrm{data}\\
        \Psi(T, \cdot) \widehat{\Psi}(T, \cdot) = p_\mathrm{prior}
    \end{array}
\end{equation}
where $\Psi$ and $\widehat\Psi$ can also be seen as solutions to the Schrödinger system of equations (we recall that the trace of the Hessian is the Laplace operator). $f$ and $g$ are functions defined on the same space as above.

If $\Psi$ and $\widehat\Psi$ are solutions of these PDEs, then the solution of the SB problem is the path measure (i.e, that is in $\mathcal{P}(p_\mathrm{data}, p_\mathrm{prior})$ of the following SDEs:
\begin{equation}
    \label{eq:sb_sdes}
\begin{aligned}
    \dint\mathbf{X}_t & = \bracks{f+g^2\nabla_x\log\Psi(t,\mathbf{X}_t)}\dint t + g\dint \mathbf{W}_t, \quad \mathbf{X}_0 \sim p_\mathrm{data}\\
    \dint \mathbf{X}_t & = \bracks{f-g^2\nabla_x\log\widehat\Psi(t,\mathbf{X}_t)}\dint t + g \dint \mathbf{W}_t, \quad \mathbf{X}_T \sim p_\mathrm{prior}
\end{aligned}
\end{equation}
This establishes a first link between PDEs founding the SB problem and SDEs characterizing the SGM model.
Those SDEs are really similar to the ones of the SGM model, except for the terms $\nabla_x\log\Psi(t,\mathbf{X}_t)$ which adds non-linearity to the drift and $\nabla_x\log\widehat\Psi(t,\mathbf{X}_t)$ which can be seen as a modified score function.
In this situation, we have to learn both the score function and the drift function, while SGM only learns the score.

% The whole idea is to get rid of this term "$\nabla_x\log\widehat\Psi(t,\mathbf{X}_t)$" by adapting the ELBO loss.
To create sample using this method, we have to solve the PDEs \eqref{eq:sb_pdes} to obtain $\widehat\Psi$.
However, those are hard to solve, the proposed method of FBSDEs (Forward-Backward SDEs) is shown to be a convenient way to link the optimality conditions of the SB problem to the ELBO maximization of the likelihood $\log_0^\mathrm{SGM}(\mathbf{x}_0)$.

\section{Proposed Method}
There are two main methods to solve PDEs: linearization-based method and sampling-based method. The latter is a consequence of the Feynman-Kac lemma that provides a solution to parabolic PDEs with a conditional expectation (suitable for sampling), which establishes a link between this kind of PDE with stochastic processes. Here, the Nonlinear Feynman-Kac formula enables to link parabolic PDEs to forward-backward coupled SDEs as below. First let's introduce a more general framework, so that (1) and (2) fall inside:

\begin{equation}\label{eq:fbsde1}
\left\{\begin{array}{ll}
   \dint\mathbf{X}_t  & = f(t,\mathbf{X}_t)\dint t + G(t,\mathbf{X}_t)\dint\mathbf{W}_t, \quad \mathbf{X}_0 =\mathbf{x}_0  \\
   \dint\mathbf{Y}_t & = - h(t,\mathbf{X}_t, \mathbf{Y}_t,\mathbf{Z}_t)\dint t + \mathbf{Z}(t,\mathbf{X}_t)^\top \dint\mathbf{W}_t,\quad \mathbf{Y}_T = \phi(\mathbf{X}_T).
\end{array}\right.
\end{equation}

Then, with the \textbf{Itô Formula} \cite{EXARCHOS2018159}, it is shown that if $\mathbf{v} \equiv v(t,\mathbf{X}_t)$ and $G(t,\mathbf{X}_t)$ are solution of the following PDE:
\begin{equation}\label{eq:fbpde1}
\frac{\partial v}{\partial t} + \frac{1}{2}\mathrm{Tr}(\nabla_\mathbf{x}^2 vGG^\top) + \nabla_\mathbf{x}v^\top f + h(t,\mathbf{x},v,G^\top\nabla_\mathbf{x}v) = 0, \quad v(T,\mathbf{x}) = \phi(\mathbf{x})
\end{equation}
Then we get :
$$
v(t,\mathbf{X}_t) = \mathbf{Y}_t \quad \text{and} \quad G(t,\mathbf{X}_t)^\top \nabla_\mathbf{x}v(t,\mathbf{X}_t) = \mathbf{Z}_t
$$
Where $\mathbf{Y}_t$ and $\mathbf{Z}_t$ are variables from (\ref{eq:fbsde1}). Therefore, it is possible to solve one of the two problem \ref{eq:fbsde1} or \ref{eq:fbpde1} to solve the other. The equation (\ref{eq:fbpde1}) can be viewed as a second-order parabolic PDE, which can be solved using FBSDE (\citeauthor{pmlr-v120-pereira20a} \citeyear{pmlr-v120-pereira20a}). Similarly, instead of solving the PDEs linked to the SB problem, it is possible to solve the underlying forward and backward SDEs. But some adaptations are necessary. Then, a new process is defined:
\begin{equation}\label{eq:newFBSDE}
\def\arraystretch{2.2}
\left\{\begin{array}{ll}
\dint\mathbf{X}_t & = (f+g\mathbf{Z}_t) \dint t + g \dint \mathbf{W}_t  \\
  \dint\mathbf{Y}_t & = \frac{1}{2}\mathbf{Z}_t^\top\mathbf{Z}_t\dint t + \mathbf{Z}_t^\top\dint\mathbf{W}_t \\
  \dint\widehat{\mathbf{Y}}_t & = \pars{\frac{1}{2}\widehat{\mathbf{Z}}_t^\top\widehat{\mathbf{Z}}_t + \nabla_\mathbf{x} . (g\widehat{\mathbf{Z}}_t - f) + \widehat{\mathbf{Z}}_t^\top\widehat{\mathbf{Z}}_t} \dint t + \widehat{\mathbf{Z}}_t^\top\dint\mathbf{W}_t
\end{array}\right.
\end{equation}
Where, under the same PDE as in (\ref{eq:fbpde1}), where $\mathbf{v}$ is replaced with $\log \Psi$ and $\log\widehat\Psi$. It is then possible to use two times the \textbf{Itô formula} with $\dint \log \Psi$ and $\dint \log\widehat\Psi$. Therefore we have:
$$
\begin{aligned}
    \mathbf{Y}_t = \log\Psi(t,\mathbf{X}_t), \quad \mathbf{Z}_t = g \nabla_\mathbf{x}\log \Psi(t,\mathbf{X}_t)\\
    \widehat{\mathbf{Y}}_t = \log\widehat\Psi(t,\mathbf{X}_t), \quad \widehat{\mathbf{Z}}_t = g\nabla_\mathbf{x}\log\widehat\Psi(t,\mathbf{X}_t)
\end{aligned}
$$
Then, it is easy to check that $\mathbf{Y}_t + \widehat{\mathbf{Y}}_t = \log(\Psi(t,\mathbf{X}_t)\widehat\Psi(t,\mathbf{X}_t)) = \log p_t^\mathrm{SB}(\mathbf{X}_t)$ according to \ref{eq:SB-PDE} constraints.

We can also notice that if we have an approximation of $\mathbf{Z}_t$ and $\widehat{\mathbf{Z}}_t$, $g$ being known (by design), we also have an approximation of $\nabla_x\log\widehat\Psi(t,\mathbf{X}_t)$ and $\nabla_x\log\Psi(t,\mathbf{X}_t)$. $\mathbf{Z}_t$ and $\widehat{\mathbf{Z}}_t$ can be understood as policy to learn in order that $p_T$ reaches $p_\mathrm{prior}$, which is imposed in the SB model but not in SGM (where the \textit{prior} distribution is approximated). Therefore $\mathbf{Z}_t$ force the forward process to reach the prior distribution while $\widehat{\mathbf{Z}}_t$ makes adjustments accordingly in the backward process.

Now, using the fact that:

\begin{align}
\esp{\mathbf{Y}_0 + \widehat{\mathbf{Y}}_0|\mathbf{X}_0 = x} & =  \nabla_x\log\widehat\Psi(t,\mathbf{x})\\
\esp{\mathbf{Y}_0|\mathbf{X}_0 = x} &= \esp{\mathbf{Y}_T - \int_0^T \dint \mathbf{Y}_t \bigg\rvert\mathbf{X}_0 = x}\\
\esp{\mathbf{Y}_0|\mathbf{X}_0 = x} & = \esp{\mathbf{Y}_T|\mathbf{X}_0 = x} -\int_0^T \pars{\esp{\int_0^T \frac{1}{2}\|\mathbf{Z}_t\|²\bigg\rvert\mathbf{X}_0 = x} + \esp{\mathbf{Z}_t \dint \mathbf{W}_t|\mathbf{X}_0 = x}}\dint t \\
\esp{\widehat{\mathbf{Y}}_0|\mathbf{X}_0 = x} & = \esp{\widehat{\mathbf{Y}}_T|\mathbf{X}_0 = x} - \int_0^T \esp{\frac{1}{2}\|\widehat{\mathbf{Z}}_t\|^2 + \nabla_x \cdot (g\widehat{\mathbf{Z}}_t - f) + \widehat{\mathbf{Z}}_t^\top \mathbf{Z}_t}\dint t
\end{align}


This allows them to derive the theorem that solves all our issues:
\begin{equation}\label{eq:logpSB}
\begin{aligned}
\log p_0^\mathrm{SB}(\mathbf{x}_0) & = \esp{\log p_T(\mathbf{X}_T)}- \int_0^T \esp{\frac{1}{2}\|\mathbf{Z}_t\|^2 + \frac{1}{2}\|\widehat{\mathbf{Z}}_t\|^2 + \nabla_x \cdot (g\widehat{\mathbf{Z}}_t - f) + \widehat{\mathbf{Z}}_t^\top \mathbf{Z}_t} \dint t
\end{aligned}
\end{equation}

We now have an ELBO to maximize with every term tractable.
That allows us to use a trainable network to learn the optimal controls $\mathbf{Z}_t$ and $\widehat{\mathbf{Z}}_t$ only, that are the optimal forward and backward drifts for SB.

\section{Going further : SB FB-SDE to solve PDEs}
It is possible to train neural network in order to sample paths that solve partial differential equations. In this part, some applications will be presented using the FBSDEs method to solve complexe PDEs.

%\subsection{Quelques exemples de domaine où la résolution de PDEs est essentielle}

A famous example is the Hamilton-Jacobi-Bellman (HJB) PDE. It is often used in optimal control theory since its solution gives the sought value function for an optimal control problem, and it plays a crucial role in dynamic programming providing efficient methods solving variational problems. The HJB PDE can be written as follows:

$$
\begin{aligned}
&\frac{\partial V(t,x)}{\partial t} + \min_u\braces{\frac{\partial V(t,x)}{\partial x}\times F(x,u) + C(x,u)} = 0\\
&\mathrm{s.t.} \quad V(T,x) = \phi(x(T))
\end{aligned}
$$
Where $\phi(.)$ is the function that gives the terminal state cost, $C(.)$ is a quadratic cost function and and $F(x(t),u(t)) = \dot{x}(t),\, \forall t \in [0,T]$. For instance, we can take the mathematical framework of \citeauthor{pereira2019deep} \citeyear{pereira2019deep} where the stochastic optimal control (SOC) problem is the following:
\begin{equation}\label{eq:SOC-HJB}\tag{SOC-HJB}
    J(t,\mathbf{x}(t);\mathbf{u}(t)) = \esp[\Q]{\int_t^T\pars{q(s,\mathbf{x}(s)) + \frac{1}{2}\mathbf{u}(s)^\top\mathbf{R}\mathbf{u}(s)}\dint s + \phi(\mathbf{x}(T)}
\end{equation}
Where
$$
\left\{\begin{array}{ll}
  \dint\mathbf{x}(t) & = (\mathbf{f}(t,\mathbf{x}(t)) + \mathbf{G}(t,\mathbf{x}(t))\mathbf{u}(t))\dint t + \sigma \mathbf{G}(t,\mathbf{x}(t))\mathbf{u}(t)\dint v(t) + \mathbf{\Sigma}(t,\mathbf{x}(t))\dint \mathbf{w}(t) \\
    \mathbf{x}(0) & = \xi
\end{array}\right.
$$
and where $\mathbf{w}$ is a vector of mutually independent one dimensional variables following a standard Brownian motion (hence the probability measure $\Q$). The aim is to find the function value of the \ref{eq:SOC-HJB} problem, meaning:
$$
\left\{\begin{array}{ll}
 V(t,\mathbf{x}(t))  & = \inf_{\mathbf{u}(t)\in\mathcal{U}([t,T])} J(t,\mathbf{x}(t);\mathbf{u}(t))  \\
  V(T,\mathbf{x}(T)) & = \phi(\mathbf{x}(T))
\end{array}\right.
$$
Finally, we get the following partial differential equation:
\begin{equation}\label{eq:HJB-PDE}\tag{PDE-HJB}
    \left\{\begin{array}{ll}
     \frac{\partial V(t,\mathbf{x})}{\partial t} + q + \nabla_\mathbf{x}V(t,\mathbf{x})^\top \mathbf{f} - \frac{1}{2}\nabla_\mathbf{x}V(t,\mathbf{x})^\top \mathbf{G}\hat{\mathbf{R}}^{-1}\mathbf{G}^\top \nabla_\mathbf{x}V(t,\mathbf{x}) + \frac{1}{2}\mathrm{Tr}(\nabla_\mathbf{x}^2 V(t,\mathbf{x})\mathbf{\Sigma}\mathbf{\Sigma}^\top) = 0 \\
      V(T,\mathbf{x}(T))  = \phi(\mathbf{x}(T))
    \end{array}\right.
\end{equation}
Where $\hat{\mathbf{R}} := \mathbf{R} + \sigma^2 \mathbf{G}^\top \nabla_\mathbf{x}^2 V(t,\mathbf{x})$ and the optimal control is $\mathbf{u}^\star(t,\mathbf{x}) =  - \hat{\mathbf{R}}^{-1}\mathbf{G}(t,\mathbf{x})^\top \nabla_\mathbf{x}V(t,\mathbf{x})$. Then, \ref{eq:HJB-PDE} is a parabolic PDE and can be solve using the FBSDE method as above. As well as before, the Itô formula gives a link between this PDE and forward-backward SDEs: TODO
%%%%%%%%%%%%%%%%%%%%%%%%%%%%%%%%%%%%%%%%%%%%%%%%%%%%%%%%%%%

\section{From Schrodinger to SDE: j'écris juste là pour éviter de casser tes trucs, mais globalement on se retrouve à la fin pour fusionner}

The Schrödinger Bridge problem, a fundamental topic in probability theory and stochastic analysis, involves finding a probability distribution that connects two given distributions at different time points. In this research project, we focus on solving this problem using the Forward-Backward Stochastic Differential Equations (FB-SDE) framework. Specifically, we compare two formulations: one based on diffusion processes and the other derived from the Benamou-Brenier formulation of the Schrödinger Bridge.

In this equation (named equation (1)):
\begin{equation}\label{eq:sgm1}
\begin{aligned}
\dint{\mathbf{X}_t} & = f(t,\mathbf{X}_t)\dint{t} + g(t) \dint{\mathbf{W}_t}, \quad \mathbf{X}_0 \sim p_\mathrm{data}\\
(\text{time-reversal}) \quad \dint{\mathbf{X}_t} & = \bracks{f(t,\mathbf{X}_t) - g(t)^2\nabla_x\log p_t(\mathbf{X}_t)}\dint{t} + g(t) \dint{\mathbf{W}_t}, \quad \mathbf{X}_T \sim p_T
\end{aligned}
\end{equation}
the SDE associated with the diffusion process is presented. It describes the evolution of $\mathbf{X}_t$ from an initial distribution $p\mathrm{data}$ to a final distribution $p_T$ at time $T$. Notably, the time-reversal version of Equation (1) provides insights into the backward dynamics of the process.

On the other hand, this equation (named equation (2)):
$$
\begin{aligned}
    \dint\mathbf{X}_t & = \bracks{f+g^2\nabla_x\log\Psi(t,\mathbf{X}_t)}\dint t + g\dint \mathbf{W}_t, \quad \mathbf{X}_0 \sim p_\mathrm{data}\\
    \dint \mathbf{X}_t & = \bracks{f-g^2\nabla_x\log\widehat\Psi(t,\mathbf{X}_t)}\dint t + g \dint \mathbf{W}_t, \quad \mathbf{X}_T \sim p_\mathrm{prior}
\end{aligned}
$$

corresponds to the SDE derived from the Benamou-Brenier formulation of the Schrödinger Bridge. It introduces two different potentials, $\Psi(t,\mathbf{X}_t)$ and $\widehat\Psi(t,\mathbf{X}t)$, which influence the evolution of $\mathbf{X}t$. Theyfollow the following PDEs:

\begin{equation}\label{eq:SB-PDE}\tag{SB-PDE}
\begin{aligned}
&\left\{\begin{array}{cc}
  \frac{\partial\Psi}{\partial t}   & = -\nabla_x\Psi^\top f - \frac{1}{2}\mathrm{Tr}(g^2\nabla_x^2\Psi)  \\
     \frac{\partial\widehat{\Psi}}{\partial t} & = -\nabla_x.(\widehat{\Psi}f) + \frac{1}{2}\mathrm{Tr}(g^2\nabla_x^2\widehat{\Psi})
\end{array}\right. \\
\mathrm{s.t.}\; &\Psi(0,.)\widehat\Psi(0,.) = p_\mathrm{data}, \,\Psi(T,.)\widehat\Psi(T,.) = p_\mathrm{prior}
\end{aligned}
\end{equation}

The initial and final distributions, $p\mathrm{data}$ and $p\mathrm{prior}$ respectively, play a crucial role in this formulation.

Although these two formulations exhibit similarities, there exists a fundamental distinction. In certain cases, such as linear or degenerate situations, Equation (1) admits an analytical solution for the score function. This analytical solution greatly simplifies the process of generating $X_0$ from $X_T$. However, Equation (2) requires solving a coupled system of PDEs to obtain an analytical solution.

In this project, our primary objective is to recover the score function, as it enables straightforward generation of $X_0$ from $X_T$. To achieve this, we focus on the formulation presented in Equation (4) :
\begin{equation}\label{eq:newFBSDE}
\def\arraystretch{2.2}
\left\{\begin{array}{ll}
  \dint\mathbf{X}_t & = (f+g\mathbf{Z}_t) \dint t + g \dint \mathbf{W}_t  \\
  \dint\mathbf{Y}_t & = \frac{1}{2}\mathbf{Z}_t^\top\mathbf{Z}_t\dint t + \mathbf{Z}_t^\top\dint\mathbf{W}_t \\
  \dint\widehat{\mathbf{Y}}_t & = \pars{\frac{1}{2}\widehat{\mathbf{Z}}_t^\top\widehat{\mathbf{Z}}_t + \nabla_\mathbf{x} . (g\widehat{\mathbf{Z}}_t - f) + \widehat{\mathbf{Z}}_t^\top\widehat{\mathbf{Z}}_t} \dint t + \widehat{\mathbf{Z}}_t^\top\dint\mathbf{W}_t
\end{array}\right.
\end{equation}

By knowing $Z_t$ and $\hat{Z}_t$, we can determine the score function and effectively solve the Schrödinger Bridge problem.

To learn $Z_t$ and $\hat{Z}_t$, we adopt a parametric approach, typically employing a neural network as the function approximator. Similar to the score function, we require a tractable and effective loss function. In this regard the following loss, proves advantageous as it introduces a novel loss function derived from the likelihood expression:
\begin{equation}\label{eq:logpSB}
\begin{aligned}
\log p_0^\mathrm{SB}(x_0) & = E\left[\log p_T(X_T)\right] - \int_0^T E\left[\frac{1}{2}\|Z_t\|^2 + \frac{1}{2}\|\hat{Z}_t - g\nabla_x \log p_t^\mathrm{SB} + Z_t\|^2 - \frac{1}{2}\|g\nabla_x \log p_t^\mathrm{SB} - Z_t\|^2 - \nabla_x \cdot f\right] dt, \\
& = E\left[\log p_T(X_T)\right] - \int_0^T E\left[\frac{1}{2}\|Z_t\|^2 + \frac{1}{2}\|\hat{Z}_t\|^2 + \nabla_x \cdot (g\hat{Z}_t - f) + \hat{Z}_t^T Z_t\right] dt
\end{aligned}
\end{equation}
The key advantage of this loss function is its exclusion of the score function, thereby simplifying the learning process to solely involve $Z_t$ and $Z_t^\mathrm{hat}$.

This function is derivated similarly to the following loss used in diffusion process but also using equation (4):
\begin{equation}
\begin{aligned}
\log p_0^{\mathrm{SGM}}(x_0) & \geq \mathcal{L}_\mathrm{SGM}(x_0; \theta) = E[\log p_T(X_T)] - \int_0^T E\left[\frac{1}{2}g^2\|s_t\|^2 + \nabla_x \cdot (g^2s_t - f)\right] dt, \\
& = E[\log p_T(X_T)] - \int_0^T E\left[\frac{1}{2}g^2\left\|s_t - \nabla_x \log p_{t|x_0}\right\|^2 - \frac{1}{2}\left\|g\nabla_x \log p_{t|x_0}\right\|^2 - \nabla_x \cdot f\right] dt,
\end{aligned}
\end{equation}









\newpage

% Bibliographie
\newpage
\nocite{*}
\addcontentsline{toc}{section}{Reference}
\bibliography{bibliography}



%%%%%%%%%%%%%%%%%%%%%%%%%%%%%%%%%%%%%%%%%%%%%%%%%%%%%%%%%%%%

\newpage
\appendix

\end{document}
